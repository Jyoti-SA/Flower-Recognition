\documentclass[paper=a4, fontsize=11pt]{article} % A4 paper and 11pt font size
\usepackage[a4paper, margin=1.3in]{geometry}
% ---- Entrada y salida de texto -----

\usepackage[T1]{fontenc} % Use 8-bit encoding that has 256 glyphs
\usepackage[utf8]{inputenc}
% \usepackage[light,math]{iwona}

\usepackage{fancyhdr}
\usepackage{fancybox}
\usepackage{pseudocode}


% ---- Idioma --------

\usepackage[spanish, es-tabla]{babel} % Selecciona el español para palabras introducidas automáticamente, p.ej. "septiembre" en la fecha y especifica que se use la palabra Tabla en vez de Cuadro

% ---- Otros paquetes ----

\usepackage{amsmath,amsfonts,amsthm} % Math packages
\usepackage{graphics,graphicx, floatrow} %para incluir imágenes y notas en las imágenes
\usepackage{graphics,graphicx, float} %para incluir imágenes y colocarlas
\usepackage{enumerate}
\usepackage{subfigure}
% \makesavenoteenv{tabular}
% \makesavenoteenv{table}
% Para hacer tablas comlejas
%\usepackage{multirow}
%\usepackage{threeparttable}
\usepackage{sectsty} % Allows customizing section commands
\allsectionsfont{\centering \scshape} % Make all sections centered, the default font and small caps

\usepackage{fancyhdr} % Custom headers and footers
\usepackage[usenames, dvipsnames]{color}
\usepackage{colortbl}
% \usepackage{minted}
\usepackage{xcolor}
\usepackage{url}
\usepackage{cancel}

% \newmintedfile[mycpp]{c++}{
%     linenos,
%     numbersep=5pt,
%     gobble=0,
%     frame=lines,
%     framesep=2mm,
% }

% \newmintedfile[myc]{c}{
%     linenos,
%     numbersep=5pt,
%     gobble=0,
%     frame=lines,
%     framesep=2mm,
% }

% \newmintedfile[mypython]{python}{
%     linenos,
%     numbersep=5pt,
%     gobble=0,
%     frame=lines,
%     framesep=2mm,
% }

\usepackage{cite}

\usepackage[bookmarks=true,
    bookmarksnumbered=false, % true means bookmarks in
             % left window are numbered
    bookmarksopen=false,   % true means only level 1
             % are displayed.
    colorlinks=true,
    urlcolor=webblue,
    citecolor=webred,
    linkcolor=webblue]{hyperref}
\definecolor{webgreen}{rgb}{0, 0.5, 0} % less intense green
\definecolor{webblue}{rgb}{0, 0, 0.5}  % less intense blue
\definecolor{webred}{rgb}{0.5, 0, 0} % less intense red

%% Define a new 'leo' style for the package that will use a smaller font.
\makeatletter
\def\url@leostyle{%
  \@ifundefined{selectfont}{\def\UrlFont{\sf}}{\def\UrlFont{\small\ttfamily}}}
\makeatother
%% Now actually use the newly defined style.
\urlstyle{leo}

\pagestyle{fancyplain} % Makes all pages in the document conform to the custom headers and footers
\fancyhead{} % No page header - if you want one, create it in the same way as the footers below
\fancyfoot[L]{} % Empty left footer
\fancyfoot[C]{} % Empty center footer
\fancyfoot[R]{\thepage} % Page numbering for right footer
\renewcommand{\headrulewidth}{0pt} % Remove header underlines
\renewcommand{\footrulewidth}{0pt} % Remove footer underlines
\setlength{\headheight}{13.6pt} % Customize the height of the header

\numberwithin{equation}{section} % Number equations within sections (i.e. 1.1, 1.2, 2.1, 2.2 instead of 1, 2, 3, 4)
\numberwithin{figure}{section} % Number figures within sections (i.e. 1.1, 1.2, 2.1, 2.2 instead of 1, 2, 3, 4)
\numberwithin{table}{section} % Number tables within sections (i.e. 1.1, 1.2, 2.1, 2.2 instead of 1, 2, 3, 4)

\setlength\parindent{0pt} % Removes all indentation from paragraphs - comment this line for an assignment with lots of text

\newcommand{\horrule}[1]{\rule{\linewidth}{#1}} % Create horizontal rule command with 1 argument of height

%%%%% Para cambiar el tipo de letra en el título de la sección %%%%%%%%%%%
% \usepackage{sectsty}
% \chapterfont{\fontfamily{pag}\selectfont} %% for chapter if you want
% \sectionfont{\fontfamily{pag}\selectfont}
% \subsectionfont{\fontfamily{pag}\selectfont}
% \subsubsectionfont{\fontfamily{pag}\selectfont}

%----------------------------------------------------------------------------------------
% TÍTULO Y DATOS DEL ALUMNO
%----------------------------------------------------------------------------------------

\title{
\normalfont \normalsize
\textsc{{\bf Visión por Computador (2016-2017)} \\ Grado en Ingeniería Informática \\ Universidad de Granada} \\ [25pt] % Your university, school and/or department name(s)
\horrule{0.5pt} \\[0.4cm] % Thin top horizontal rule
\huge Proyecto Final\\Reconocimiento de Flores\\% The assignment title
\horrule{2pt} \\[0.5cm] % Thick bottom horizontal rule
}

\author{Marta Gómez Macías\\Correo: mgmacias95@correo.ugr.es\\
Braulio Vargas López\\Correo: brauliovarlop@correo.ugr.es} % Nombre y apellidos

\date{\normalsize\today} % Incluye la fecha actual

%----------------------------------------------------------------------------------------
% DOCUMENTO
%----------------------------------------------------------------------------------------

\begin{document}

\maketitle % Muestra el Título
\pagenumbering{gobble}
\newpage %inserta un salto de página

\tableofcontents % para generar el índice de contenidos
\newpage
\pagenumbering{arabic}

\section{Definición del problema}

El problema elegido consiste en extraer una serie de características del conjunto de imágenes para ser capaces de reconocer la especie de flor que es de las 17 especies posibles, mediante una imagen. Este problema consta de varias partes:

\begin{enumerate}
    \item \textbf{Extracción de las características principales de la imagen}.
    \item \textbf{Entrenar un algoritmo de aprendizaje a partir de las características extraídas}.
\end{enumerate}

Para esto, vamos a hacer uso de distintos descriptores, como son el descriptor \textit{HOG}, \textit{SIFT}, entre otros. Una vez tengamos los descriptores, utilizaremos la técnica \textit{Bag of Words} para obtener un descriptor que nos sirva para obtener un modelo de aprendizaje, capaz de clasificar las distintas clases de flores.

La técncia \textit{Bag Of Words} consiste en extraer una serie de características de las imágenes, conocidas como palabras. Cada imagen, tendrá una serie de palabras que la identifican y con estas palabras, podremos construir un vocabulario que almacena todas las palabras extraídas de todas las imágenes.

Con este vocabulario, podremos hacer un histograma que nos indica con qué fuerza vota o no una imagen a dichas palabras del vocabulario, con lo que obtendremos un vector de características de esa imagen, útil para el algoritmo de aprendizaje, y que no deja de tener relación con la propia imagen.

Este vocabulario lo podemos construir haciendo uso de técnicas que se centran en la geometría de la imagen y son invariantes a la escena, como son los descriptores de SIFT, SURF, etc. Pero, también es posible utilizar otro tipo de descritores como son el ya mencionado, HOG.

El descriptor HOG, conocido como \textit{\textbf{Histogram of Oriented Gradientes}} es muy utilizado para la detección de objetos. Este descriptor computa un histograma de valores donde se indican la frecuencia con la que se dan las direcciones de los gradientes en pequeñas divisiones de la imagen.

\section{Enfoque de la implementacion y eficiencia}

La implementación de este trabajo se ha realizado entera en Python, usando la librería OpenCV para la parte de visión por computador, y la librería \textit{scikit-learn}\cite{scikit-learn}.

\subsection{Extracción de las características principales de la imagen}

En primer lugar, hemos generado una función capaz de cargar todo el conjunto de datos de la carpeta \textit{Dataset} en memoria para poder comenzar a trabajar con las imágenes.

A continuación, para poder extraer las características de una imagen, primero hemos implementado una función que calcula varios tipos de descriptores de una imagen, según el tipo de detector que queramos usar. Según el tipo de descriptor, generará un detector que irá construyendo un vocabulario, a partir de los \textit{KeyPoints} detectados y su descriptor.
Este vocabulario se inicializa como un \texttt{BOWKMeansTrainer} y con la función \texttt{add} vamos añadiendo los descriptores obtenidos.

\subsection{Clusterizar los descriptores obtenidos}

Para obtener los clusters, al haber creado el vocabulario con BOWKMeansTrainer, podemos usar la función cluster para realizar el clustering con K-Medias y así obtener el vocabulario clusterizado, listo para obtener los descriptores de cada imagen.
Esta función de OpenCV, ya implementa el método de clustering utilizando multiprocesamiento en paralelo, para aumentar la eficiencia del algoritmo.

\subsection{Obtener los histogramas de cada imagen}

Una vez construida la bolsa de palabras, tenemos ver con qué fuerza vota cada una de las imágenes para las distintas palabras que hemos extraído. Para ello, utilizaremos la función de OpenCV \texttt{BOWImgDescriptorExtractor}, que necesita un detector de puntos, que crearemos del mismo tipo que se usó para la construcción del vocabulario, y un \textit{matcher}. En este caso, usaremos un \textit{matcher} basado en fuerza fuerza bruta.
Con todo esto, podremos calcular el histograma para cada imagen y obtener el descriptor para cada una de ellas, haciendo uso de la función de OpenCV compute que utiliza la imagen y el detector para obtener los \textit{KeyPoints} y computar el histograma.

\bibliography{citations} %archivo citas.bib que contiene las entradas 
\bibliographystyle{plain} % hay varias formas de citar

\end{document}